\documentclass[12 pt]{article}

\begin{document}

\section{Introduction}
Digital healthcare is the way of the future in healthcare. Healthcare in the next generation will be more advanced than it is now. There will be no need to go to the hospital in the future.Heartbeat, blood pressure, and sugar levels are all routinely checked.Healthcare will become more portable and computerised in the future.In this section, we look at the future of healthcare and what it might look like in the future.whatever equipment will be important in the future to make
Healthcare is simple, economical, time-saving, and life-saving.more than that We also look into the different types of issues that have arisen.People in the healthcare field will confront challenges in the future.


\section{Future Healthcare Innovations}


\subsection{Digital Data and Information}
Errors in medicine administration are being reduced. Software can detect anomalies between a patient's health and prescriptions and alert health providers and patients to a potential drug error by analysing medical records.
Assisting in the prevention of disease. A substantial number of repeat patients, sometimes known as frequent fliers, visit emergency rooms. Big data analysis can aid in the identification of this sort of patient as well as the development of preventative programmes to deter them from returning.
More precise staffing. Predictive analytics could help hospitals and clinics better plan staffing by predicting admission rates.

\subsection{Telehealth}

Telehealth refers to the use of digital information and communication technology, such as computers and mobile devices, to access and manage health care services remotely. These could be technologies that you use at home or that your doctor employs to improve or supplement health-care services. Upload meal logs, prescriptions, dosage, and blood sugar readings to a nurse who answers electronically via a mobile phone or other device. People who live in rural or isolated regions should have access to health services. Assist people in self-managing their health care.


\subsection{Artificial Intelligence}

Artificial intelligence has the ability to dramatically transform healthcare. AI algorithms can mine medical records, build treatment plans, and create medications more faster than any other actor in the healthcare ecosystem, including doctors.

Atomwise employs supercomputers to search through a library of molecular structures for remedies. The start-up initiated a virtual search in 2015 for safe, current drugs that may be adapted to treat Ebola. They discovered two medications that were anticipated by the company's artificial intelligence algorithm that might considerably reduce Ebola infectivity.





\subsection{Virtual Reality}
Virtual reality (VR) is the name given to a technology that allows a user to use a VR headset to replicate a situation or experience of interest in a computer-generated environment. The simulation is immersive, thus special 3-D goggles with a screen or gloves with sensory feedback may be required to assist the user learn from experience in this virtual world. Medical training, patient treatment, pain management, physical therapy, and rehabilation are some of the applications.



\subsection{Nanotechnology}
Nanotechnology is concerned with the atomic and molecular engineering of systems. It combines molecular chemistry, physics, and engineering to obtain an edge over the distinctive changes in material properties that occur at the nanoscale.
Nanomedicine is a branch of medicine that uses nanotechnology to treat and diagnose diseases using nanoparticles in medical devices, as well as nanoelectronic biosensors and molecular nanotechnology. Nanomedicine is currently being utilised in the development of smart pills as well as the treatment of cancer.







\end{document}